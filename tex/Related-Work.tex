\section{Related Work}\label{relatedWork}
In 2013, authors from University of California published their research on \textit{discretized streams} (D-Streams)~\cite{SparkDStream}. They implemented their work as part of Apache Spark. As already mentioned, data is collected into so called \textit{micro batches}, which means that several tuples in a pre-defined, sub-second-sized window get grouped together into an immutable data set. A stream of \textit{micro batches} is called a D-Stream. Each \textit{micro batch} is then distributed to a worker node within a cluster by a master node. The worker performs a series of operations on the dataset, thus creating a new immutable data set. In this way, each operation is stateless. 

This system provides some interesting advantages to the continuous stream processing model of Aurora (and its later iterations, i.e. Borealis and StreamBase). Regarding recovery, the system is able to track the lineage of operations, that were performed on each \textit{micro batch}. This is possible, because of a data structure called \textit{Resilient Distributed Datasets} (RDD), that stores the transformations that created the data set. Because the intermittent results are stored in-memory and supersede the replication of data for fault tolerance, recovering times are sped up significantly. Additionally, also because replication is not necessary, the end-to-end processing times are increased drastically. 

Beside having a faster and more resilient performance than systems with continuous operator systems, D-Streams have one downside. By intentionally delaying the processing of arriving tuples, the minimum end-to-end processing delay is raised. This makes micro batching not suitable for applications with sub-second response requirements, such as some stock trading applications. This issue is adjustable to some degree, because users are able to define the window for creating micro batches by themselves.

The D-Stream approach makes use of earlier research on stream processing, and subsequently focuses on the weaknesses of earlier systems, such as recovering mechanisms and parallel execution. The performance increases of using D-Streams instead of other systems are remarkable, which partly explains the ongoing popularity of Apache Spark~\cite{SparkUsers}.