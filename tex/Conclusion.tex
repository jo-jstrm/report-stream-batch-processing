\section{Conclusion}\label{conclusion}
With the amount of digital data drastically increasing, new use cases and business models connected to handling data have arisen. Two ways of analyzing data stand out: the processing of continuous data flows, so called streams, and the processing of large batches of data at once. Traditional RDBMSs were not designed for and therefore show poor performance in those tasks. In this report we presented two systems that address the issues with processing big data. Aurora is a data stream management system from 2003 which presents fundamentally different approaches to RDBMSs with regards to how data streams are handled. The main feature is the boxes-and-arrows system that is able to perform analysis of continuous streams, as well as views and ad-hoc queries, in real time. It has, like Stratosphere, fundamentally different assumptions from relational database management systems about the data that it processes. Stratosphere, on the other hand, focuses on batch processing. The system's main features include in-situ data processing and an extensible operator set with access to different abstraction layers, which is a generalization of MapReduce~\cite{MapReduce} by offering more operators and native support for iterative operations. It also features automated optimization and parallel, distributed execution. In our opinion, both systems were successful, because they made the right assumptions about the kind of data processing tasks that they wanted to address, and consistently followed these assumptions in their respective system design.