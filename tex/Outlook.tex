\section{Outlook}\label{outlook}
Aurora, as part of the Borealis project that added support for parallel and distributed execution, was commercialized as StreamBase Inc. in 2003 and was the first commercial real-time stream processing engine~\cite{AuroraBorealis2016}. The company provides an enterprise data stream management system and was bought by TIBCO in 2013~\cite{TIBCObuysStreamBase}, which continues to offer enterprise solutions for streaming applications.~\footnote{\url{https://www.tibco.com}. Last accessed on 09.07.2019}

Members of the Stratosphere project continued with publications until 2016. With the transition to Apache Flink, the project now supports stream and batch processing. The team followed their research goals, which we described in Chapter~\ref{stratosphereImprovement}, and mainly focused on parallelization and optimization topics~\cite{StratospherePublications}. Parallel to the academic Stratosphere project, some of the authors founded ``data artisans", the first commercial installment that offers an enterprise solution based on Apache Flink, and continues to contribute to Flink's open source development. Data artisans was renamed to Ververica~\cite{VervericaAbout} and sold to Alibaba in January 2019~\cite{VervericaAlibaba}. These developments and the large industrial user base show that the Stratosphere project was hugely successful.

In the process of our work on stream and batch processing, we noticed an interesting dichotomy of processing data as a stream and as a batch. Aurora, a data stream management system, generates small batches of data – tuple trains - before processing them. The developers call the principle \textit{train scheduling}. Further, the system tries to pipeline execution of each tuple train over as many operators as possible, which is called \textit{superbox scheduling}. This means that the self-declared streaming platform relies on small batches of data for efficient processing. 

Stratosphere, on the other hand, was developed as a batch processing system. Interestingly, Stratosphere's execution engine aims to process each data tuple as soon as it arrives, which can lead to some tuples being several operations ahead of other tuples from the same data set. Therefore Stratosphere decomposes a batch and, at least on each Task Manager, creates a pipelined stream out of it.

There is a third processing principle regarding big data, called micro-batching, which treats arriving data as small batches and then processes those small batches together. This technique is employed by popular Apache Spark. In addition, Apache Flink allows stream- as well as batch processing with the same engine.

Following those observations, it seems that for big data applications, a combination of treating data situationally as a stream- and a batch, is the key for a high performing system. Incoming batches may be partly formed into a stream and vice versa.